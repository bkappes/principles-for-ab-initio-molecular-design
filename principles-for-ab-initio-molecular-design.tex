%%%%%%%%%%%%%%%%%%%%%%%%%%%%%%%%%%%%%%%%%%%%%%%%%%%%%%%%%%%%%%%%%%%%%
%% This is a (brief) model paper using the achemso class
%% The document class accepts keyval options, which should include
%% the target journal and optionally the manuscript type.
%%%%%%%%%%%%%%%%%%%%%%%%%%%%%%%%%%%%%%%%%%%%%%%%%%%%%%%%%%%%%%%%%%%%%
\documentclass[journal=jpccck,manuscript=article]{achemso}

%%%%%%%%%%%%%%%%%%%%%%%%%%%%%%%%%%%%%%%%%%%%%%%%%%%%%%%%%%%%%%%%%%%%%
%% Place any additional packages needed here.  Only include packages
%% which are essential, to avoid problems later. Do NOT use any
%% packages which require e-TeX (for example etoolbox): the e-TeX
%% extensions are not currently available on the ACS conversion
%% servers.
%%%%%%%%%%%%%%%%%%%%%%%%%%%%%%%%%%%%%%%%%%%%%%%%%%%%%%%%%%%%%%%%%%%%%
\usepackage[version=3]{mhchem} % Formula subscripts using \ce{}

%%%%%%%%%%%%%%%%%%%%%%%%%%%%%%%%%%%%%%%%%%%%%%%%%%%%%%%%%%%%%%%%%%%%%
%% If issues arise when submitting your manuscript, you may want to
%% un-comment the next line.  This provides information on the
%% version of every file you have used.
%%%%%%%%%%%%%%%%%%%%%%%%%%%%%%%%%%%%%%%%%%%%%%%%%%%%%%%%%%%%%%%%%%%%%
%%\listfiles

%%%%%%%%%%%%%%%%%%%%%%%%%%%%%%%%%%%%%%%%%%%%%%%%%%%%%%%%%%%%%%%%%%%%%
%% Place any additional macros here.  Please use \newcommand* where
%% possible, and avoid layout-changing macros (which are not used
%% when typesetting).
%%%%%%%%%%%%%%%%%%%%%%%%%%%%%%%%%%%%%%%%%%%%%%%%%%%%%%%%%%%%%%%%%%%%%
\newcommand*\mycommand[1]{\texttt{\emph{#1}}}

%%%%%%%%%%%%%%%%%%%%%%%%%%%%%%%%%%%%%%%%%%%%%%%%%%%%%%%%%%%%%%%%%%%%%
%% Meta-data block
%% ---------------
%% Each author should be given as a separate \author command.
%%
%% Corresponding authors should have an e-mail given after the author
%% name as an \email command. Phone and fax numbers can be given
%% using \phone and \fax, respectively; this information is optional.
%%
%% The affiliation of authors is given after the authors; each
%% \affiliation command applies to all preceding authors not already
%% assigned an affiliation.
%%
%% The affiliation takes an option argument for the short name.  This
%% will typically be something like "University of Somewhere".
%%
%% The \altaffiliation macro should be used for new address, etc.
%% On the other hand, \alsoaffiliation is used on a per author basis
%% when authors are associated with multiple institutions.
%%%%%%%%%%%%%%%%%%%%%%%%%%%%%%%%%%%%%%%%%%%%%%%%%%%%%%%%%%%%%%%%%%%%%
\author{Glen Alen Ferguson}
	\altaffiliation{Equally contributing author}
\affiliation[NREL-1]
	{National Bioenergy Center, %
	 National Renewable Energy Laboratory, %
	 15013 Denver West Parkway, Golden, CO}

\author{Branden B. Kappes}
	\altaffiliation{Equally contributing author}
%\email{s.k.laborator@bigpharma.co}
\affiliation[CSM]
	{Dept. of Mechanical Engineering, %
	Colorado School of Mines, %
	1500 Illinois St., Golden, CO 80401}

\author{James Whitaker}
\author{Daniel Ruddy}
\affiliation[NREL-1]
	{National Bioenergy Center, %
	 National Renewable Energy Laboratory, %
	 15013 Denver West Parkway, Golden, CO}

\author{Jeremy Neubauer}
\author{Shriram Santhanagopalan}
\email{Shriram.Santhanagopalan@nrel.gov}
\affiliation[NREL-2]
	{Transportation \& Hydrogen Systems, %
	 National Renewable Energy Laboratory, %
	 15013 Denver West Parkway, Golden, CO}
%%%%%%%%%%%%%%%%%%%%%%%%%%%%%%%%%%%%%%%%%%%%%%%%%%%%%%%%%%%%%%%%%%%%%
%% The document title should be given as usual. Some journals require
%% a running title from the author: this should be supplied as an
%% optional argument to \title.
%%%%%%%%%%%%%%%%%%%%%%%%%%%%%%%%%%%%%%%%%%%%%%%%%%%%%%%%%%%%%%%%%%%%%
\title[Principles for Ab Initio Molecular Design]
	{Ab initio design principles for property optimization of energy storage molecules: Predicting the energy density in redox reactions for small molecules}

%%%%%%%%%%%%%%%%%%%%%%%%%%%%%%%%%%%%%%%%%%%%%%%%%%%%%%%%%%%%%%%%%%%%%
%% Some journals require a list of abbreviations or keywords to be
%% supplied. These should be set up here, and will be printed after
%% the title and author information, if needed.
%%%%%%%%%%%%%%%%%%%%%%%%%%%%%%%%%%%%%%%%%%%%%%%%%%%%%%%%%%%%%%%%%%%%%
\abbreviations{}
\keywords{DFT, Gaussian, oxidation, reduction, electrochemistry, flow battery, electrochemistry}

\begin{document}
%%%%%%%%%%%%%%%%%%%%%%%%%%%%%%%%%%%%%%%%%%%%%%%%%%%%%%%%%%%%%%%%%%%%%
%% The manuscript does not need to include \maketitle, which is
%% executed automatically.  The document should begin with an
%% abstract, if appropriate.  If one is given and should not be, the
%% contents will be gobbled.
%%%%%%%%%%%%%%%%%%%%%%%%%%%%%%%%%%%%%%%%%%%%%%%%%%%%%%%%%%%%%%%%%%%%%
\begin{abstract}
  TBW
\end{abstract}

%%%%%%%%%%%%%%%%%%%%%%%%%%%%%%%%%%%%%%%%%%%%%%%%%%%%%%%%%%%%%%%%%%%%%
%% Start the main part of the manuscript here.
%%%%%%%%%%%%%%%%%%%%%%%%%%%%%%%%%%%%%%%%%%%%%%%%%%%%%%%%%%%%%%%%%%%%%

\section{Introduction}
\paragraph{Ab initio design}
% Difficult
% Prediction from direct calculations

\paragraph{Prediction from groups of molecules}
% Matt�s paper
% Lei�s paper
% Hutchinson�s papers
% Kitchin�s papers
% N�rskov�s papers

\paragraph{Predicting molecular properties is difficult but important}
% Correlations require large datasets of relatively inexpensive quantum calculations where the dataset is expensive to calculate
% Correlations are often true on for a specific domain of chemistry but are not general
% Properties are often a function of many variables
%% Substructures 
%%% Number of constituents
%%% Identity of constituents
%%% Relative substituent positions

\paragraph{Most large datasets are currently used to screen based on heuristics}
% Screening can find candidates but gives no information about what improves properties
% While the desired properties are usually known the mapping of calculated values is properties is often unknown and system as designed ad hoc to produce the desired data

\paragraph{We have a framework for mapping molecular properties to functional groups and substructures (Fingerprints)}

\paragraph{Case study: Liquid and Semi-liquid batteries}
% A possible solution for grid and EV energy storage 
% Require molecules with precise properties
%% Electrochemical window
%% Low MW
%% Good solubility
% Properties can be difficult and expensive to measure

\section{Computational methods}
\paragraph{Creating the dataset}
% Calculation of properties
%% Global optimization using FF
%% Quantum optimization
%% Solvation energy
%% Electrochemical window
%% Other properties

\paragraph{Using the dataset for screening}
% Properties can be enumerated 
% Plotting simple properties
%% Substituted values plotted against changes in properties
%%% Functional groups (F, Cl, CN, CF3, CH3, t-butyl, OCH3)
%%% Properties (Solvation, 1st and 2nd Redox Potentials)
%%% 3D plot of all of these
%% Quantification of changes due to functional group addition

\paragraph{Infrastructure for making prediction from the dataset}
% Fingerprinting of molecules in the dataset
%% Conversion to bit strings
% Measuring differences from a reference using bit strings
%% Tanimoto
% Clustering of fingerprints vs. properties (Solvation, 1st and 2nd Redox Potentials)
%% K-means of 3
% Classification
%% Property vectors 
%% Bit strings for FP
% Mapping substring contributions to a property

\section{Results and Discussion}
\paragraph{Property calculations}
% Experimental theoretical correlation for redox potentials
%Cite previous work
% Description of property changes upon functionalization

\paragraph{Solvation}
% Changes
%% Number of constituents
%% Identity of constituents
%% Relative substituent positions
% Quantitative model of the effect of functionalization on solvation

\paragraph{Redox potentials}
% Changes based on the 
%% Number of constituents
%% Identity of constituents
%% Relative substituent positions
% Quantitative model of the effect of functionalization on redox potentials
% How to improve anodes and cathodes

\section{Conclusions}

%%%%%%%%%%%%%%%%%%%%%%%%%%%%%%%%%%%%%%%%%%%%%%%%%%%%%%%%%%%%%%%%%%%%%
%% The "Acknowledgement" section can be given in all manuscript
%% classes.  This should be given within the "acknowledgement"
%% environment, which will make the correct section or running title.
%%%%%%%%%%%%%%%%%%%%%%%%%%%%%%%%%%%%%%%%%%%%%%%%%%%%%%%%%%%%%%%%%%%%%
\begin{acknowledgement}

The authors thank NEXUS, the National Renewable Energy Laboratory's high-performance computing center, for providing computational resources. Funding through the Defense Advanced Research Project Agency Grant No. 1234-56789 is gratefully acknowledged.

\end{acknowledgement}

%%%%%%%%%%%%%%%%%%%%%%%%%%%%%%%%%%%%%%%%%%%%%%%%%%%%%%%%%%%%%%%%%%%%%
%% The appropriate \bibliography command should be placed here.
%% Notice that the class file automatically sets \bibliographystyle
%% and also names the section correctly.
%%%%%%%%%%%%%%%%%%%%%%%%%%%%%%%%%%%%%%%%%%%%%%%%%%%%%%%%%%%%%%%%%%%%%
%\bibliography{achemso-demo}

\end{document}
